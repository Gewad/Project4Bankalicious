\documentclass{article}

% Package to make about everything dutch
\usepackage[dutch]{babel}

\title{Opdracht C:\\
	Individuele Plan Bijdrage Groepsdeel}
\author{Bytegroep 10\\
	Paul Wondel}

\begin{document}

\maketitle
\newpage

\tableofcontents

\newpage

\section{Team SWOT}

Wij hebben positieve en negatieve punten opgesteld over hoe wij denken dat het project gaat verlopen.
Wij hebben zowel interne als externe punten gedocumenteerd.

\vspace{5mm}

\begin{table}[h!]
\caption{Team SWOT: Negatieve punten}
\label{tab: Negatieve punten}
\begin{tabular}{l|l}
        & \textbf{Negatief}\\
        \hline
        {Intern} 	& Werk wordt niet altijd gelijk verdeeld \\
			& Wij stellen vaak deadlines uit \\
			& Niet iedereen neemt initiatief \\
			& Groepsbesprekingen zijn vaak rommelig \\
	\hline
        {Extern}	& Documentatie kan halverwege veranderd worden\\
\end{tabular}
\end{table}

\begin{table}[h!]
\caption{Team SWOT: Positieve punten}
\label{tab: Positieve punten}
\begin{tabular}{l|l}
        & \textbf{Positief} \\
        \hline
        {Intern} 	& Goede werk sfeer \\ 
			& Wij staan klaar voor elkaar \\
			& Mensen nemen initiatief waar zij dat kunnen \\
			& Wij beginnen vroeg met het project \\
			& Onze documentatie is goed terug te vinden \\
	\hline
        {Extern}	& Wij krijgen een leerzame excursie naar een bank \\
			& Dit project is er duidelijke documentatie \\	
\end{tabular}
\end{table}

\newpage

\section{Individuele SWOT}
\begin{table}[h!]
\caption{SWOT: Positieve Punten}
\label{tab: Positief}
\begin{tabular}{|l|l|}
	& \textbf{Positief}\\
	\hline
			& \textit{Strengths}\\
	{Intern}	& Goed communiceren met collegas en medestudenten \\
			& Doorzettings vermogen	\\
			& Werk hard bij aangewezen taken \\
			& Flexibel \\
			& Bereid mensen of groepsleden bij te helpen \\
			& Samenwerking \\
			& Samenstudie \\
	\hline
			& \textit{Opportunities}\\
	{Extern}	& Kansen om te leren onbekende systemen te omzeilen \\
			& Ervaring in het omgaan met onbekende systemen \\
			& in samenwerking met mijn eigen systeem \\
			& Leren om zelf algoritmen op te stellen \\
\end{tabular}
\end{table}

\begin{table}[h!]
\caption{SWOT: Negatieve punten}
\label{tab: Negatief}
\begin{tabular}{|l|l|}
	& \textbf{Negatief}\\
	\hline
			& \textit{Weaknesses}\\
	{Intern}	& Zelf bedenken van Algoritmen \\
			& Weinig voorkennis over coderen \\
			& Neem niet altijd het initiatief in de groep \\
			& Te lang doen over sommige taken \\
			& Weinig ervaring in het bouwen van electronica \\
	\hline
			& \textit{Threats}\\
	{Extern}	& Fysieke bouw van de ATM \\
			& Algoritme bedenken voor de Centrale Bank \\
\end{tabular}
\end{table}

\newpage

\section{Taakverdeling}

Hieronder is onze taakverdeling te zien.

\begin{itemize}
  \item \textbf{Paul H.}  Organizator
  \item \textbf{Paul W.}  Verbinder 
  \item \textbf{Gerard} Expert 
  \item \textbf{Merijn} Document Manager, Onderzoeker
  \item \textbf{Aron} Tester
  \item \textbf{Boas} Kwaliteitsbewaker 
  \item \textbf{Floor} Designer, expert
  \item \textbf{Mohammed} Designer
\end{itemize}

\vspace{10mm}
\section{Rol: de "Verbinder"}
Ik heb de rol van de verbinder aangenomen.
De verbinder zorgt ervoor dat er een goede
samenwerking en communicatie is tussen zijn
eigen groep en de andere groepen.
Het is mijn taak om met de andere groepen afspraken
te maken om onze systemen samen met elkaar
te laten werken.
Ik hoor ook elke keer met de andere groepen
te controleren als de afspraken worden nagekomen.
Mochten er complicaties oplopen,
dan bespreek ik dit met de andere groepen
om nieuwe afspraken te kunnen maken.

\end{document}
