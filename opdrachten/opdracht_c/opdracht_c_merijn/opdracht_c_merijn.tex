\documentclass{article}

% Package to make about everything dutch
\usepackage[dutch]{babel}

\title{Opdracht C}
\author{Merijn Plagge \\ 0941200}

\begin{document}

\maketitle

\section{Team SWOT}

Wij hebben positieve en negatieve punten opgesteld over hoe wij denken dat het project gaat verlopen.
Wij hebben zowel interne als externe punten gedocumenteerd.

\begin{table}[h!]
\caption{Team SWOT: Positief}
\label{tab: Team SWOT: Positief}
\begin{tabular}{l|l}
        & \textbf{Positief} \\
        \hline
        {Intern} 	& Goede werk sfeer \\ 
			& Wij staan klaar voor elkaar \\
			& Mensen nemen initiatief waar zij dat kunnen \\
			& Wij beginnen vroeg met het project \\
			& Onze documentatie is goed terug te vinden \\
        {Extern}	& Wij krijgen een leerzame excursie naar een bank \\
			& Dit project is er duidelijke documentatie \\	
\end{tabular}
\end{table}

\begin{table}[h!]
\caption{Team SWOT: Negatieve punten}
\label{tab: Negatieve punten}
\begin{tabular}{l|l}
        & \textbf{Negatief}\\
        \hline
        {Intern} 	& Werk wordt niet altijd gelijk verdeeld \\
			& Wij stellen vaak deadlines uit \\
			& Niet iedereen neemt initiatief \\
			& Groepsbesprekingen zijn vaak rommelig \\
        {Extern}	& Documentatie kan halverwege veranderd worden\\
\end{tabular}
\end{table}


\newpage

\section{Taakverdeling}

Hieronder is onze taakverdeling te zien.

\begin{itemize}
  \item \textbf{Paul H.} Organizator
  \item \textbf{Paul W.} Verbinder 
  \item \textbf{Gerard} Expert 
  \item \textbf{Merijn} Documentatie manager, onderzoeker
  \item \textbf{Aron} Tester
  \item \textbf{Boas} Kwaliteitsbewaker 
  \item \textbf{Floor} Designer, expert
  \item \textbf{Mohammed} Designer
\end{itemize}

\section{Mijn SWOT}

Hieronder is mijn SWOT te lezen.
Ik heb het hier over de onderdelen waar ik zelf moeite mee ga hebben dit project.
Ook heb ik het hier over de onderdelen waar ik juist erg positief over ben zelf.

\begin{table}[h!]
\caption{Mijn SWOT: Positief}
\label{tab: Mijn SWOT: Positief}
\begin{tabular}{l|l}
        & \textbf{Positief} \\
        \hline
        {Intern} 	& Ik streef naar kwaliteit \\ 
			& Ik wil graag duidelijkheid cre\"eren \\
			& Ik begin bij projecten vroeg en werk vooruit \\
			& Ik doe graag onderzoek over nieuwe onderwerpen \\
        {Extern}	& Ik zie de mogelijkheid om over goede beveiliging te leren \\ 
			& Ik denk veel te leren op het gebied van adviseren \\
\end{tabular}
\end{table}

\begin{table}[h!]
\caption{Mijn SWOT: Negatief}
\label{tab: Mijn SWOT: Negatief}
\begin{tabular}{l|l}
        & \textbf{Negatief} \\
        \hline
        {Intern} 	& Ik moet veel inhalen van periode 3 en heb dus weinig tijd \\ 
			& Ik ben niet zo goed ik plannen. \\
        {Extern}	& Het kan zijn dat mijn groepsgenoten dingen uitstellen \\ 
			& Er is niet genoeg communicatie tussen mijn groepsgenoten \\
			& Het is niet duidelijk hoe ik met Git moeten werken \\
\end{tabular}
\end{table}

\newpage

\section{Mijn rol}

Ik ben van plan om mijn rollen uitgebreid uit te werken.
Ook ben ik van plan op hier twee extra punten mee te halen.
Deze rollen zal ik het hele project door uit moeten voeren.

\subsection{Document manager}

De afgelopen projecten heb ik altijd de taak op mij genomen op te zorgen dat documenten netjes online staan.
Zelf ben ik nog niet echt blij met het resultaat en ik wil het dit project een stuk netter gaan doen.
Ik wil meer met \emph{Git} en \emph{branching} gaan werken omdat dit een stuk meer structuur geeft dan \emph{Google Drive}, wat we nu gebruiken.
Ik moet zorgen dat iedereen weet moet er met deze software gewerkt en ik moet zorgen dat Git juist gebruikt wordt.

\subsection{Onderzoeker}
Ik wil dan graag zorgen dan nieuwe informatie beschikbaar is waar wij dat nodig hebben.
Ik heb een goed overzicht over welke informatie nog niet duidelijk ik, wil hier graag achteraan gaan. 
Ook heb ik al een kennis snack aangevraagd over Git en Version control omdat dit niet duidelijk is.
Ik ben dus al bezig met deze rol.
Dit soort dingen zal ik op mij moeten nemen.

\end{document}
