\documentclass{article}

\usepackage[dutch]{babel}

\begin{document}

\title{Advies Centrale Bank}
\author{Paul Wondel}
\date{\today}
\maketitle

\section{abstract}
In opdracht voor project 4 heb ik dit verslag opgesteld.
In dit verslag worden er de kwaliteitseisen en functionaliteitseisen
beschreven voor de inrichting van de Centrale Bank.

\clearpage
\newpage

\tableofcontents

\clearpage
\newpage

\section{Centrale Bank}
De Centrale Bank kan fungeren als een zogenaamd tussenpersoon.
De Centrale Bank is verbonden met alle lokale banken.
Als een lokale bank informatie nodig heeft van een andere lokale bank,
kan hij de gegevens opvragen aan de centrale bank.
De centrale bank vervolgens vraagt aan de andere lokale bank.
En die stuurt de gegevens naar de centrale bank waardoor de centrale bank
de gegevens terug kan sturen de lokale bank die de gegevens had aangevraagd.\\

Elke lokale bank heeft een database waarin alle gegevens van hun klanten in staan.
Deze databases staan op een server van de lokale bank (dus een lokale server).



\section{Server Communicatie}
 

\subsection{Sockets}

\subsubsection{Java Applicatie}

\subsection{Message Protocols}

\subsubsection{MQ: Message Queuing}
Berichten worden door verschillende programmas gebruikt
om data met elkaar te delen tussen een zender en ontvanger.
Programmas gebruiken veel berichten na elkaar maar niet
alle berichtn kunnen tegelijk verwerkt worden.
Daarom is er een queue. Een gueue is een lijst met wachtende onderwerpen.
Message Queuing zorgt ervoor dat de messages in een volgorde verstuurd worden.

Message Queuing is een asynchroon communicatie protocol.
Dat houdt in dat het systeem een bericht in een queue lijst plaatst
en dat hoeft niet per direct verwerkt te worden.
Email is een voorbeeld van asynchroon communicatie.

Bij message queuing wordt er gebruik gemaakt van decoupling.
Dit is om de afzender en ontvanger te van elkaar te scheiden.
Decoupling is een proces dat delen van een systeem die afhankelijk
zijn van elkaar scheidt en zelfstandig maakt.

 


\subsubsection{MQTT: Message Queuing Telemetry Transport}
MQTT heeft hetzelfde principe als MQ, maar MQTT is een protocol en MQ
is een manier van werken.


\section{Advies}

\section{Kwaliteitseisen}

\subsection{Security}
Encryptie
Firewall
Arduino beveiliging





\end{document}
