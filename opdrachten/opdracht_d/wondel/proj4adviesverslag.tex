\documentclass{article}

\usepackage[dutch]{babel}

\begin{document}

\title{Advies Centrale Bank}
\author{Paul Wondel}
\date{\today}
\maketitle

\begin{abstract}
In opdracht voor project 4 heb ik dit verslag opgesteld.
In dit verslag worden er de kwaliteitseisen en functionaliteitseisen
beschreven voor de inrichting van de Centrale Bank.
\end{abstract}

\clearpage
\newpage

\tableofcontents

\clearpage
\newpage

\section{Centrale Bank}
De Centrale Bank kan fungeren als een zogenaamd tussenpersoon.
De Centrale Bank is verbonden met alle lokale banken.
Als een lokale bank informatie nodig heeft van een andere lokale bank,
kan hij de gegevens opvragen aan de centrale bank.
De centrale bank vervolgens vraagt aan de andere lokale bank.
En die stuurt de gegevens naar de centrale bank waardoor de centrale bank
de gegevens terug kan sturen de lokale bank die de gegevens had aangevraagd.\\

Elke lokale bank heeft een database waarin alle gegevens van hun klanten in staan.
Deze databases staan op een server van de lokale bank (dus een lokale server).



\section{Server Communicatie}
 

\subsection{Sockets}

\subsubsection{Java Applicatie}

\subsection{Message Protocols}

\subsubsection{MQ: Message Queuing}
Berichten worden door verschillende programmas gebruikt
om data met elkaar te delen tussen een zender en ontvanger.
Programmas gebruiken veel berichten na elkaar maar niet
alle berichtn kunnen tegelijk verwerkt worden.
Daarom is er een queue. Een gueue is een lijst met wachtende onderwerpen.
Message Queuing zorgt ervoor dat de messages in een volgorde verstuurd worden.

Message Queuing is een asynchroon communicatie protocol.
Dat houdt in dat het systeem een bericht in een queue lijst plaatst
en dat hoeft niet per direct verwerkt te worden.
Email is een voorbeeld van asynchroon communicatie.

Bij message queuing wordt er gebruik gemaakt van decoupling.
Dit is om de afzender en ontvanger te van elkaar te scheiden.
Decoupling is een proces dat delen van een systeem die afhankelijk
zijn van elkaar scheidt en zelfstandig maakt.

Message queuing werkt als volgt.
Een message producer(een applicatie bv) maakt een message aan met data erin.
De message stuurt hij dan naar een 'Message Broker'.
Een Message Broker houdt de message queue bij.
Wanneer de producer al zijn messages naar de message broker
gestuurd heeft stuurt de message broker de message queue naar
de message consumer.
De message consumer kan nu een voor een de messages in de message queue
verwerken en hoeft zo niet alle messages tegelijk te behandelen.

MQ brengt vele voordelen met zicht mee.
MQ heeft redundantie via persistentie om data te blijven behouden.
Je kan batches met messages zodat niet elke message appart wordt
verstuurd, maar dat je in 1 keer een aantal messages stuurt.
Dit zorgt voor de efficientie.
MQ kan ook in verschillende talen worden geschreven.



%Schrijven over MQ dat toepasselijk is op de centrale bank
Op dit moment stuurt de client van de lokale bank berichten na mekaar steeds naar de lokale server.
Er is geen regeling of controle bij het sturen van de berichten.
Op kleine schaal is dat niet van belang.
Op een grotere schaal is komt dit wel in conflict met de efficientie van
de server connectie tussen de centrale bank en de lokale banken.
De centrale bank is verbonden met tientallen servers van de lokale banken.
De lokale banken sturen ook constant berichten naar de centrale bank.
De centrale bank kan dit niet in een keer verwerken.
Dus om dit probleem op te lossen maken we gebruik van MQ.
Message Queuing zorgt er voor dat de berichten(met queries voor de database)
die per client gestuurd worden naar de centrale bank eerst in een lijst worden gedaan
om in een keer gestuurd te worden naar de centrale bank.
Dit zorgt ook voor minder data belast.
De server kan dan gerust per queue de berichten een voor een verwerken.


\subsubsection{MQTT: Message Queuing Telemetry Transport}
MQTT is het protocol dat gebruikt wordt om MQ (Message Queuing) toe te passen in een netwerk.


%Schrijven hoe MQTT toepasselijk is op de centrale bank


\section{Advies}
%Wat advieseer je aan de mensen om te doen bij het maken van de bank?


\section{Kwaliteitseisen}
%Welke kwaliteitseisen zijn er gesteld voor je de centrale bank en hoe pak je ze aan?


\subsection{Security}

\paragraph{Encryptie}
Als we het hebben over encryptie,
dan hebben wij het gewoon over het versleutelen van data.
SSL en Hashing zijn hier goede voorbeelden van.


\paragraph{Firewall}
\paragraph{Arduino beveiliging}





\end{document}
